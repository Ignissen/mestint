\documentclass[a4paper, 12pt, fullpage]{article}
\usepackage[utf8]{inputenc}
\usepackage[magyar]{babel}
\usepackage[margin=1.5cm,includefoot,footskip=30pt,]{geometry}
\usepackage{amsmath}

\author{PJT}
\title{Mestint 2. előadás}

\begin{document}
    \maketitle
    \newpage
    \textbf{Példán keresztüli magyarázat:}\\
    $A = \{\begin{bmatrix}
        A_{11} & & \\
        & & \\
        & & A_{33}\\
    \end{bmatrix} \subseteq VALAMI\}$\\
    $C \subseteq A$\\
    $\sigma = \{f: fel, l: le, b: balra, j: jobbra\}$\\
    $le: dom(le) \rightarrow A$\\
    $dom(le) = (\begin{bmatrix}
        A_{11} & &\\
        &&\\
        &&A_{33}\\
    \end{bmatrix} \subseteq A | A_{31}, A_{32}, A_{33} \neq 0\rbrace$\\
    $le(\begin{bmatrix}
        A_{11}&&\\
        &&\\
        &&A_{33}\\
    \end{bmatrix}) = \begin{bmatrix}
        A_{11}&&\\
        &&\\
        &&A_{33}\\
    \end{bmatrix}$\\
    \textbf{Fogalmak:}\\
    \begin{itemize}
        \item \textbf{Közvetlen elérhetőség:} Legyen $a \in A, b \in A$. Az $a$ állapotból a $b$ állapot közvetlenül elérhető, ha van olyan operátort tudunk alkalmazni az $a$ állapotra és célállapota a $b$ állapot.\\Jelölés: $a \Rightarrow b$
        \item \textbf{Elérhetőség:} Legyen $a,b \in A$. Az $a$ állapotból $b$ állapot elérhető, ha ezek\begin{itemize}
            \item $a = b$
            \item Van olyan $o_1, o_2, ..., o_k$ operátorsorozat, $o_n \in o, i=1,2,...,k, i > 1$, amelyre $a_i \Rightarrow a_{i+1} \Rightarrow ... \Rightarrow b$\\Jelölés: $a \Rightarrow^* b$
        \end{itemize}
    \end{itemize}
    $<A, kezdo, C, o>$ megoldható, ha a kezdőállapotból elérhető valamely célállapot.\\
    $C \neq \emptyset$\\
    \begin{center}
        \Large{\textbf{Ágensszemlélet}}
    \end{center}
    \textbf{Ágens:} Bármi lehet, aminek van \begin{itemize}
        \item Érzékelője/Szenzora: információt nyerhetnek a környezetről
        \item Beavatkozója/Aktuátor: valamilyen módosítást hajtanak végre a környezeten
    \end{itemize}
    Érzékelés $\longrightarrow$ cselekvés?\\
    Ágensffüggvény érzékeléssorozathoz 1 darab cselekvést rendel.\\
    \textbf{Determinisztikus regresszió:} Determinisztikus környezetben a környezet állapota meghatározható a környezet előzetes állapotából és az ágens által végrehajtott cselekvésből.\\
    \textbf{epizodikus(sorozatszerű) környezetek:} Egy adott cselekvéssorozat a következő cselekvéssorozatra nincs hatással.
    \textbf{Diszkrét/folytonos környezetek:}
    \begin{itemize}
        \item Diszkrét: Véges kiértékelés, véges cselekvés
        \item Folytonos: Robot taxi példa
    \end{itemize}
    \textbf{Egy-, vagy többágenses környezet:}
    \begin{itemize}
        \item Egyágenses környezet: pl. kirakó kirakása
        \item Többágenses: pl. robotporszívó\begin{itemize}
            \item Kooperatív környezet: pl. automata járművek
            \item Kompetetív környezet: pl. parkolási "harc"
        \end{itemize}
    \end{itemize}
\end{document}